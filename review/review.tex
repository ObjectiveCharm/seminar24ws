\documentclass[a4paper,9pt]{scrartcl}
\usepackage{amssymb, amsmath} % needed for math
\usepackage[utf8]{inputenc}   % this is needed for umlauts
\usepackage[USenglish]{babel} % this is needed for umlauts
\usepackage[T1]{fontenc}      % this is needed for correct output of umlauts in pdf
\usepackage[margin=2.5cm]{geometry} %layout
\usepackage{hyperref}         % hyperlinks
\usepackage{color}
\usepackage{framed}
\usepackage{enumerate, enumitem}  % for advanced numbering of lists
\usepackage[autostyle,english=american]{csquotes}
\MakeOuterQuote{"} % for enquote

\newcommand\titletext{Peer-Review of\\"Advancing European Railway Systems: A Transition to Unified Control with ERTMS, ETCS, and 5G"}

\title{\titletext}
\author{Guanru Chen}

\hypersetup{
  pdfauthor   = {Guanru Chen},
  pdfkeywords = {peer review},
  pdftitle    = {Lineare Algebra}
}

\usepackage{microtype}

\begin{document}
\maketitle
\section{Introduction}
This is the peer review of the seminar paper \textit{Advancing European Railway Systems: A Transition to Unified Control with ERTMS, ETCS, and 5G}, which discusses the railway control system adopted by European railways and the prospect of 5G's practical usage in this domain. It also lists multiple implementations by country together with figures. It also covers the unified standards within the European Union and their evolution, such as the adoption of new mobile network standards. The last section of the paper shows the potential future of the combination and integration of 5G technology in European railway systems and the associated topics (improvement, feasibility, and security)

\section{Summary of the Content}
\subsection{Section 1. Introduction}
It briefly introduces the idea and development of several unified standards (ERTMS and ETCS) and gives an overview of the deployment of mobile networks in railway systems.
\subsection{Background information}
It lists some typical railway controlling standards and mechanisms with technical data. It also makes comparisons while taking real-world production factors into consideration.
\begin{enumerate}[label=(\Alph*)]
    \item Germany
    \begin{enumerate}[label=(\Roman*),ref=(\Alph{enumi}-\Roman*)]
        \item The Punktförmige Zugbeeinflussung (PZB) Standard
        \item The Linienzugbeeinflussung/Linienförmige Zugbeeinflussung (LZB) Standard
    \end{enumerate}
    \item France
    \begin{enumerate}[label=(\Roman*),ref=(\Alph{enumi}-\Roman*)]
        \item Trackside signaling
        \item Transmission Voie-Machine (TVM)
    \end{enumerate}
    \item Finland
    \begin{enumerate}[label=(\Roman*),ref=(\Alph{enumi}-\Roman*)]
        \item Junakulunvalvonta (JKV)
    \end{enumerate}
\end{enumerate}

\subsection{ERTMS and ETCS}
It introduces the concept of ERTMS and ETCS and shows some technical details with figures. And discuss the current status of the deployment of mobile network technologies, especially the LTE standard in these control systems.
The last paragraph of this section proposes the most significant idea, adopting 5G for railway systems. It discusses its feasibility, prospects, and advantages. Besides it mentions the security and compatibility concerns. It shows a real-world example with Hamburg's train station and surrounding areas.

\subsection{Conclusion}
This section concludes the prospects of the former topics and shows the prospects for the adoption of new 5G technologies.


\section{Overall Feedback}
It proposed a very innovative and practical topic combining computer network and telecom technologies with a real-world production scenario.
But I think the contents of 5G topic in the last section need to be ordered and rearranged into multiple subsections, or it should be extracted to a standalone section.
And there are more topics worth discussing, such as migrating to the new 5G technology-based control system, infrastructure requirements, and software support.
It is also remarkable that the authors are good at English writing, and it has almost no grammar mistakes (much better than I do).
I think it should focus more on 5G technologies, not only on showing details of previous railway control systems that are not even so digitalized (but I don't mean it should be completely eliminated for comparison).
In terms of organization and structure, it lacks a comparison of related works.

\section{Major Remarks}
\begin{itemize}
    \item \enquote{designated as ATP-VR/RHK, which is derived from Alstom’s
EBICab 900 technology.} - need citation.
\end{itemize}

\section{Minor Remarks}


\end{document}