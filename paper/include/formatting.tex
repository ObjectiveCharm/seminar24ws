\section{Writing papers with \LaTeX}\label{cheiThu4}
This section provides a small introduction to writing papers with \LaTeX{}.
Uniquely label your sections\,/\,subsections, figures or equations.
You can afterwards reference labeled items.
Consider sections, figures and tables as proper names, i.\,e., capitalize them
when referencing, e.\,g.\ Section~\ref{cheiThu4}.
Equations such as
\begin{align}
	\nonumber G(f) &= \frac{1}{\sqrt{2\pi}}\int_{-\infty}^{\infty}
	g(t)e^{-j\omega ft} \operatorname dt\\
	&= \frac{1}{\sqrt{2\pi}}\int_{-\infty}^{\infty} g(t) \left (\cos(2\pi f t)
	-j\sin(2\pi ft) \right ) \operatorname dt
	\label{lai6Ohxo}
\end{align}
are labeled line by line and referenced by \eqref{lai6Ohxo}.
Equation~\eqref{lai6Ohxo} is typeset in an align environment, which allows to
align multiple lines at the same column.


\subsection{Units}
The \texttt{siunitx} package is your best friend when it comes to units.
It knows virtually any kind of unit such as \si{\kilo\meter}, \si{henry},
\si{\deci\bel}, and many more.
You can also use it to consistently typeset values, e.\,g.\ $\SI{1024}{\byte} =
\SI{1}{\kibi\byte} = \SI{1.024}{\kilo\byte}$.
You can even write something like $\SI{1000}{packets\per\second}$, and it can
be used in both text and math modes.

\subsection{Tables and figures}
Format your tables as shown in Table~\ref{shu1ioKo}.
\begin{table}
	\centering
	\caption{Example for a nice table}
	\label{shu1ioKo}
	\begin{tabular}{lllll}
		\toprule
		\textbf{SI prefix} & \textbf{Power of 10} && \textbf{Binary prefix} &
		\textbf{Power of 2}\\
		\midrule
		\si{\kilo\nothing} & $10^3$ && \si{\kibi\nothing} & $2^{10}$\\
		\si{\mega\nothing} & $10^6$ && \si{\mebi\nothing} & $2^{20}$\\
		\si{\giga\nothing} & $10^9$ && \si{\gibi\nothing} & $2^{30}$\\
		\si{\tera\nothing} & $10^{12}$ && \si{\tebi\nothing} & $2^{40}$\\
		\si{\peta\nothing} & $10^{15}$ && \si{\pebi\nothing} & $2^{50}$\\
		\bottomrule
	\end{tabular}
\end{table}
In contrast to figures, tables have their caption at the top.
Figure~\ref{Geivap7D} is placed inline by specifying the \texttt{h!} modifier
in the floating environment.
\begin{figure}[h!]
	\centering
	\begin{tikzpicture}[>=latex]
		\node[client,minimum size=10mm] (1) at (0,0) {};
		\node[server,minimum size=15mm] (2) at (4,0) {};
		\draw[->,line width=1pt, bend left] (1.north) to (2.north);
		\draw[->,line width=1pt, dashed] (2.west) to (1.east);
	\end{tikzpicture}
	\caption{A figure made with TikZ}
	\label{Geivap7D}
\end{figure}
Alternatively, you can tell \LaTeX\ to place floats at the top or bottom by
using the \texttt{t} and \texttt{b} modifiers, respectively.
You can also include graphic files such as PDF, PNG and JPEG by using
\texttt{\textbackslash{}includegraphics}.
However, make sure \textbf{not} to use images with an alpha channel.
Otherwise, the text of your paper will be rasterized when printed and will look
really bad.


\subsection{Citations}\label{UoQui0le}

You find a list of references at the end of the main file of your paper.
Add additional bibentries there as needed and cite them using the
\texttt{\textbackslash{}cite{}} macro.

Verbose citations must be enclosed by quotation marks with a citation
preceding or following the quoted text.
If you write a subsection in your own words but it is closely based on some
source, a citation at the beginning or end of that text block should suffice.

``%
Please number citations consecutively within brackets~\cite{b1}.
The sentence punctuation follows the bracket~\cite{b2}.
Refer simply to the reference number, as in~\cite{b3}---do not use
'Ref.~\cite{b3}' or 'reference~\cite{b3}' except at the beginning of a
sentence: 'Reference~\cite{b3} was the first $\ldots$'.'' \cite{b8}.

When you cite other papers for the first time, please list the author's names,
e.\,g.\ ``\ldots as shown by Jacobs and Bean in~\cite{b3}\ldots''.
In case there are more than two authors, you may also write ``\ldots as shown
by Eason et al.\ in~\cite{b1}\ldots'' instead.
Avoid a line break preceding the brackets by substituting the whitespace
between the preceding word and the \texttt{\textbackslash{}cite} macro by
``\textasciitilde{}''.

Invest a significant amount of time into finding the right related work:
\begin{itemize}
	\item Prefer scientific sources, i.\,e.\ papers, over websites.
	\item Do not blindly cite anything.
	Have a close look at the authors, the date of publications, and where the
	paper was published.
	In general, cite older literature first as it is likely to be closer to the
	original source of a topic.
	\item Do not hesitate to cite a paper multiple times, or find multiple
	independent sources for the same statement.
	However, cite the papers by date of publication.
	\item You will notice that many papers appear to be published multiple
	times by the same authors with some timely distance.
	Find and cite the oldest publication only.
	An exception may be papers that later appeared in journals as an extended
	version.
\end{itemize}

Citing URLs such as \cite{b9} is problematic for two reasons:
URLs tend to be long and cryptic and---more severe---are volatile, i.\,e.\,
they tend to break without warning.
Using services such as \url{archive.org} fix the latter problem but do not make
URLs more readable.
In any case, avoid URL shorteners since they represent another point of failure
and, in addition, hide the link target.
Consequently, avoid citing URLs whenever possible.
Some more citations~\cite{b4,b6,b7,b8,b9}.


\subsection{Some common mistakes}\label{eiceFe1u}
\begin{itemize}
	\item The word ``data'' is plural, not singular.
%	\item The subscript for the permeability of vacuum $\mu_{0}$, and other
%	common scientific constants, is zero with subscript formatting, not a
%	lowercase letter ``o''.
%	\item In American English, commas, semicolons, periods, question and
%	exclamation marks are located within quotation marks only when a complete
%	thought or name is cited, such as a title or full quotation.
%	When quotation marks are used, instead of a bold or italic typeface, to
%	highlight a word or phrase, punctuation should appear outside of the
%	quotation marks.
%	A parenthetical phrase or statement at the end of a sentence is punctuated
%	outside of the closing parenthesis (like this). (A parenthetical sentence
%	is punctuated within the parentheses.)
%	\item A graph within a graph is an ``inset'', not an ``insert''. The word
%	alternatively is preferred to the word ``alternately'' (unless you really
%	mean something that alternates).
	\item Do not use the word ``essentially'' to mean ``approximately'' or
	``effectively''.
%	\item In your paper title, if the words ``that uses'' can accurately
%	replace the word ``using'', capitalize the ``u''; if not, keep using
%	lower-cased.
%	\item Be aware of the different meanings of the homophones ``affect'' and
%	``effect'', ``complement'' and ``compliment'', ``discreet'' and
%	``discrete'', ``principal'' and ``principle''.
%	\item Do not confuse ``imply'' and ``infer''.
	\item The prefix ``non'' is not a word; it should be joined to the word it
	modifies, usually without a hyphen.
	\item There is no period after the ``et'' in the Latin abbreviation ``et
	al.''.
	\item The abbreviation ``i.\,e.'' means ``that is'', and the abbreviation
	``e.\,g.'' means ``for example''.
	\item The abbreviation ``i.\,e.'' is preceded and followed by commas while
	``e.\,g.'' is only preceded by a comma.
	\item Writing $f * g$ does \emph{not} indicate a multiplication but a
	convolution, and $f \times g$ indicates a cross product.
	Chances are, you mean and ordinary multiplication, which is simply written
	as $f \cdot g$.
	\item In English, the word ``that'' is never preceded by a comma.
	Writing something like ``The door that is blue \ldots'' uniquely identifies
	a specific door, while writing ``The door, which is blue \ldots'' merely
	describes one of many doors.
	Using the word ``which'' without a preceding comma in the previous example
	would be equivalent to using ``that''.
	\item Capitalize all words except for conjunctions like ``and'',
	``or'' etc.\ in your title and citations---even if others do not.
	\item Never use short forms such as ``don't''.
	Write ``do not'' instead.
	\item Between two headlines subordinate, e.\,g.
	\texttt{\textbackslash{}section} and \texttt{\textbackslash{}subsection},
	there must be some text.
	For instance, you might briefly outline the current section.
	\item Enumerations and lists consist of at least two items, not one.
	The same holds for subsections: do not use a subsection if there is only
	one.
	\item In English, you keep writing in lower case after a colon.
	\item The word ``eventually'' has nothing to do with the German word
	``eventuell''.
	\item Subordinate clauses starting with ``because'', ``since'', etc.\ are
	separated from the main clause by a comma only if the subordinate clause
	precedes the main clause.
	\item Figures, Tables, Sections etc.\ are considered proper names, i.\,e.\
	always write ``\ldots as shown in Section~\ref{UoQui0le}''.
	Make sure to capitalize the noun.
	An exception are equations.
	Here, the noun is completely omitted except when an equation is referenced
	at the beginning of a sentence.
	See \eqref{lai6Ohxo} at the beginning of Section~\ref{cheiThu4} for an
	example.
\end{itemize}
An excellent style manual for science writers is \cite{b7}.


\subsection{How to find symbols in \LaTeX}
\emph{Detexify} \cite{b9} is an awesome website helping you with finding the
correct \LaTeX{} macro for hand-written symbols.


